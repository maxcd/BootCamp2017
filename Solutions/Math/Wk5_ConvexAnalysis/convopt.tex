\documentclass[letterpaper,12pt]{article}
\usepackage{array}
\usepackage{threeparttable}
\usepackage{geometry}
\geometry{letterpaper,tmargin=1in,bmargin=1in,lmargin=1.25in,rmargin=1.25in}
\usepackage{fancyhdr,lastpage}
\pagestyle{fancy}
\lhead{}
\chead{}
\rhead{}
\lfoot{\footnotesize\textsl{OSM Lab, Summer 2017, Math PS \#2}}
\cfoot{}
\rfoot{\footnotesize\textsl{Page \thepage\ of \pageref{LastPage}}}
\renewcommand\headrulewidth{0pt}
\renewcommand\footrulewidth{0pt}
\usepackage[format=hang,font=normalsize,labelfont=bf]{caption}
\usepackage{amsmath}
\usepackage{amssymb}
\usepackage{amsthm}
\usepackage{natbib}
\usepackage{setspace}
\usepackage{float,color}
\usepackage[pdftex]{graphicx}
\usepackage{hyperref}
\hypersetup{colorlinks,linkcolor=red,urlcolor=blue,citecolor=red}
\theoremstyle{definition}
\newtheorem{theorem}{Theorem}
\newtheorem{acknowledgement}[theorem]{Acknowledgement}
\newtheorem{algorithm}[theorem]{Algorithm}
\newtheorem{axiom}[theorem]{Axiom}
\newtheorem{case}[theorem]{Case}
\newtheorem{claim}[theorem]{Claim}
\newtheorem{conclusion}[theorem]{Conclusion}
\newtheorem{condition}[theorem]{Condition}
\newtheorem{conjecture}[theorem]{Conjecture}
\newtheorem{corollary}[theorem]{Corollary}
\newtheorem{criterion}[theorem]{Criterion}
\newtheorem{definition}[theorem]{Definition}
\newtheorem{derivation}{Derivation} % Number derivations on their own
\newtheorem{example}[theorem]{Example}
\newtheorem{exercise}[theorem]{Exercise}
\newtheorem{lemma}[theorem]{Lemma}
\newtheorem{notation}[theorem]{Notation}
\newtheorem{problem}[theorem]{Problem}
\newtheorem{proposition}{Proposition} % Number propositions on their own
\newtheorem{remark}[theorem]{Remark}
\newtheorem{solution}[theorem]{Solution}
\newtheorem{summary}[theorem]{Summary}
%\numberwithin{equation}{section}
\bibliographystyle{aer}
\newcommand\ve{\varepsilon}
\newcommand\boldline{\arrayrulewidth{1pt}\hline}

\begin{document}

\begin{flushleft}
   \textbf{\large{Math, Problem Set \#5, Convex Optimization}} \\[5pt] Instructor: Jorge Barro \\[5pt]
   Due Friday, July 21 at 8:00am
\end{flushleft}
\textbf{Homework:} 1, 2, 4, 5, 7, 13, 20, and 21 at the end of Chapter 7 of Humpherys et al. (2017) \\

\begin{itemize}
\item[7.1] Show that the convex hull of any non-empty set $S$ is convex with $S \subset V$. The hull $H=conv(S)$ is given by the set of all finite sums of the form
$\lambda_1x_1 + ... +\lambda_nx_n$ for all $x \in S, \, n \in \mathbb{N}$ with $\sum \lambda_i = 1$ and $\lambda_i \geq 0$. $H$ is convex if $\gamma y + (1-\gamma)x \in H \, \forall x, y \in H$ and $0\leq \gamma \leq1$.$
 \gamma y + (1-\gamma)x =\,  \gamma (\lambda_1y_1 + ... +\lambda_ny_n )+ (1-\gamma)(\lambda'_1x_1 + ... +\lambda'_kx_k) \, \in H$ iff $\, \gamma \sum \lambda_i + (1-\gamma) \sum \lambda'_j = 1$. Since $x, y \in H$ by definition of $H$ it follows that $\sum \lambda_i = \sum \lambda'_j = 1$. Finally,  $\gamma 1 + (1-\gamma) 1 = 1$ is true.
 
 
\item[7.2i] Claim: A hyperplane is convex. \\ Let $x_{a}$ and $x_{b}$ be any two arbitrary points in $P = \{x \in V | \langle a,x \rangle = b \}$. Then, $\lambda x_{a} + (1-\lambda) x_{b} = \lambda a_{1} x_{a1} + ... + \lambda a_{n} x_{an} + ... + (1 - \lambda)a_{1}x_{b1} + ... + (1 - \lambda)a_{n}x_{bn} = \lambda a_{1}x_{a1} + a_{1}x_{b1} - \lambda a_{1}x_{b1} + ... + \lambda_a{n}x_{an} + a_{n}x_{bn} - \lambda a_{n} x_{bn} = b + \lambda b - \lambda b = b$. Since any convex combination of the two points is still in the hyperplane $P$, we know that the hyperplane is convex. \\


\item[7.2ii] Claim: Half-spaces are convex. \\ Let $H = \{x \in V | \langle a,x \rangle  \leq b \}$ be a half-space, where $a \in V, a \neq 0,$ and $b \in {\rm I\!R}$. Then, for any two arbitrary points $x_{a}$ and $x_{b}$ in the half-space, we know that $\lambda(a_{1}x_{1} + ... + a_{n}x_{n} + (1 - \lambda)(a_{1}x_{1}^{'} + ... + a_{n}x_{n}^{'} = \lambda a_{1}x_{1} + a_{1}x_{1}^{'} - \lambda a_{1}x_{1}^{'} + ... + \lambda a_{n}x_{n} + a_{n}x_{n}^{'} - \lambda a_{n}x_{n}^{'} \leq \lambda b + b - \lambda b = b$. Since the convex combination of any two arbitrary points is in the half-space, we conclude that the half-space is convex. \\

\item[ 7.4i] Claim: $\|x- y\|^{2} = \|x-p\|^{2} + \|p-y\|^{2} + 2 \langle x-p, p-y \rangle$. \\ $\|x-y\|^{2} = \langle x-y, x-y \rangle = \langle x-p +p -y, x-p+p-y \rangle = \langle x-p, x-p \rangle + \langle x-p, p-y \rangle + \langle p-y, x-p \rangle + \langle p-y, p-y \rangle = \|x-p\|^{2} + \|p-y\|^{2} + 2\langle x-p, p-y \rangle$. \\

\item[7.4ii] $\|x-p\| 
\leq \|x-p\| + \|p-y\|$ because $\|p-y\| \geq 0$. Therefore squaring both sides, we preserve the inequality and obtain $\|x-p\|^{2} 
\leq \|x-p\|^{2} + \|p-y\|^{2} + 2 \langle x-p, p-y \rangle = \|x-y\|^{2}$. Taking the squareroot of both sides now, we obtain $\|x-p\| \leq \|x-y\|$. \\

\item[7.4iii] Given that $z = \lambda y + (1-\lambda)p$, we can write $\|x-z\|^{2} = \|x-p\|^{2} + \|p-z\|^{2} + 2\langle x-p, p-z \rangle = \|x-p\|^{2} + \|p - \lambda y - p + \lambda p\|^{2} + 2\langle x-p, p-\lambda y - p + \lambda p \rangle = \|x-p\|^{2} + \| \lambda p  - \lambda y \|^{2} + 2\langle x-p, \lambda p - \lambda y \rangle = \|x-p\|^{2} + \lambda^{2} \|p-y\| + 2\lambda \langle x-p, p-y \rangle$. \\ 

\item[ 7.4iv] Claim: If p is a projection of x onto the convex set C, then $\langle x-p, p-y \rangle \geq 0 \forall y \in C$. \\ Suppose p is the projection of $x$ onto the convex set $C$. Then we know that $\|x-z\|^2 = \|x-p\|^2 + 2\lambda \langle x-p, p-y \rangle + \lambda^2 \|y-p\|^2$. We know that the right hand side of the equation is greater than $\|x-p\|^{2}$ since p is a projection onto C and z is a point in C (z is in C because C is convex and z is a convex linear combination of points in C). Moreover, the right hand side can be rewritten as $\|x-p\|^2 + \lambda (2\langle x-p, p-y \rangle + \lambda \|y-p\|^2)$ where $\lambda \|y-p\|^2 \geq 0$. Since the expression has to be greater than or equal to $\|x-p\|^{2}$, it follows that $2\langle x-p, p-y \rangle + \lambda \|y-p\|^{2} \geq 0$ for all $y \in C$ and $\lambda \in [0,1]$. Thus, we can let $\lambda = 0$ and see that $2\langle x-p, p-y \rangle \geq 0$. \\ 

\item[ 7.6] Claim: If f is a convex function, then the set $\{x \in {\rm I\!R}^{n} | f(x) \leq c \}$ is a convex set. \\ Suppose f is a convex function. Let $x_{a}$ and $x_{b}$ be arbitrary elements of $S= \{x \in {\rm I\!R}^{n} | f(x) \leq c\}$. It remains to show that $f(\lambda x_{a} + (1-\lambda)x_{b}) \leq c$. $f(\lambda x_{a} + (1-\lambda)x_{b}) \leq \lambda f(x_{a}) + (1-\lambda)f(x_{b}) \leq \lambda c + (1-\lambda)c = \lambda c - \lambda c + c = c$, as desired. \\

\item[ 7.7] To show that f(x) is conex, we need to show that for all $x_{1}, x_{2} \in C$, $f(\mu x_{1} + (1-\mu)x_{2}) \leq \mu f(x_{1}) + (1-\mu) f(x_{2}).$ $f(\mu x_{1} + (1-\mu)x_{2}) = \sum_{i=1}^{k} \lambda_{i}f_{i}(\mu x_{1} + (1-\mu)x_{2}) \leq \sum_{i=1}^{k} \lambda_{i} [\mu f_{i}(x_{1}) + (1-\mu)f_{i}(x_{2})] = \mu \sum_{i=1}^{k} \lambda_{i}f_{i}(x_{i}) + (1-\mu)\sum_{i=1}^{k} \lambda_{i}f_{i}(x_{2}) = \mu f(x_{1}) + (1-\mu)f(x_{2})$. \\ 

\item[ 7.13] Claim: If f is convex and bounded above, then f is constant. \\ Suppose f is convex and bounded above, and suppose to the contrary that there exists $x,y$ where $f(x) \geq f(y)$. Then for any $x_{1}, x_{2} \in C$ and $0 \leq \lambda \leq 1$, we have $f(\lambda x_{1} + (1-\lambda) x_{2}) \leq \lambda f(x_{1}) + (1-\lambda)f(x_{2})$. Thus, we know $f(x) \leq  \lambda f(\frac{x-(1-\lambda)y}{\lambda}) + (1- \lambda) f(y)$, where $x = \lambda x_{1} + (1 - \lambda)x_{2}$ and $y=x_{2}$. Rearranging the inequality, we obtain $\frac{f(x) - (1-\lambda)f(y)}{\lambda} \leq f(\frac{x-(1-\lambda)y}{\lambda} \leq b$, where b is a finite upper bound for f. As $\lambda \rightarrow 0^{+}$, $\frac{f(x)-(1-\lambda)f(y)}{\lambda} \rightarrow \infty$, which contradicts the assumption that f is bounded above. Therefore, it follows that f is constant, that is $f(x) = f(y)$ for all $x,y \in C$. \\
\item[ 7.20] Claim: If f is convex and -f is also convex, then f is affine. \\ Suppose f is convex. Then, $f(\lambda x + (1-\lambda)y) \leq \lambda f(x) + (1 - \lambda) f(y)$. Suppose -f is convex. Then $-f(\lambda x + (1-\lambda)y) \leq -\lambda f(x) + -(1 - \lambda) f(y)$. Multiplying the last inequality by -1 throughout and see that $f(\lambda x + (1-\lambda)y) \geq \lambda f(x) + (1 - \lambda) f(y)$. The only way that both inequalities can be true is if $f(\lambda x + (1-\lambda)y) = \lambda f(x) + (1 - \lambda) f(y)$. Therefore f is linear and thus affine. \\ 

\item[7.21] Suppose $f(x{*}) \leq f(y) \forall y \in {\rm I\!R}^{n}$. Since $\phi$ is strictly increasing, we know that $f(x^{*})$ minimizes $\phi$ over the range of f, which means that $x^{*}$ minimizes $\phi \circ f$. Suppose $\phi \circ f(x^{*}) \leq \phi \circ f(y) \forall y \in {\rm I\!R}^{n}$. Then because $\phi$ is strictly increasing, $f(x^{*}) \leq f(y) \forall y \in C$, which means that $x^{*}$ minimizes f. 

\end{itemize}

\end{document}